\subsection{Controller}

The PI-controller is implemented in \textit{c} on the processing system in a object orientated way where each of the two controllers are defined together with their gains. When the system is booted up the controllers are initialized with the values and prepared for usage. The code for this can be seen in code sample \ref{code:pi_controller1}.

\begin{lstlisting}[style=c, caption=Initialization of PI-controller., label=code:pi_controller1]
/* Declare controllers */
Controller cQ;
Controller cD;
f32 kpQ = 1;
f32 kiQ = 1;
f32 kpD = 1;
f32 kiD = 1;

void initControllers(){
    /* Initialize controllers with gains */
    initController(&cQ, kpQ, kiQ);
    initController(&cD, kpD, kiD);
}
\end{lstlisting}

When the motor is running the only thing needed to be done is to input the reference value and the current measured value to the controller and get a new output as can be seen in code sample \ref{code:pi_controller2}. The function 'getOutput()' handles the controller gains, keeping track of the integrator part and integrator windup.

\begin{lstlisting}[style=c, caption=Usage of PI-controller., label=code:pi_controller2]
    /* Run controller */
	f32 outD = getOutput(&cD, 0, iD);
	f32 outQ = getOutput(&cQ, getTargetTorque(), iQ);
\end{lstlisting}

The implementation of the 'getOutput()' function can be seen in code sample \ref{code:pi_controller3}.
First the controller gains \textit{kp} and \textit{ki} are retrieved. Then the controller's integrator part is updated and then retrieved.

The output of the controller is found with the following equation.
\begin{equation}
    u = k_p \cdot e + k_i \cdot int
\end{equation}

Where $u$ is the output, $e$ is the error which is the target minus the input, 

$e = target - input$, $int$ is the integrator part and $k_p$ and $k_i$ are the controller gains.

The integral part of the controller might experience integrator windup because the control executing is much faster than the motor speed. To handle this a simple output limitation is implemented to limit the controller output.



\begin{lstlisting}[style=c, caption=Implementation of PI-controller 'get output'-function., label=code:pi_controller3]
/* Function to get the next output of a controller with a new input */
f32 getOutput(Controller *c, f32 target, f32 input){
	f32 error = target - input;                         // Calculate error
	
	// Collect controller data
	const f32 kp  = getKp(c);
	const f32 ki  = getKi(c);
	addToIntegrator(c, error);
	const f32 integrator = getIntegrator(c);

	f32 output =  (kp * error) + (ki * integrator);     // Calculate new output
	
	// Anti integrator windup. Limit the output to within set limits
	if(output > MAX_OUTPUT){                            // Check upper limit
		output = MAX_OUTPUT;                              // Limit output
	}
	if(output < MIN_OUTPUT){                            // Check lower limit
		output = MIN_OUTPUT;                              // Limit output
	}	
	return output;                                      // Return new output
}
\end{lstlisting}

