\subsubsection{Design}

The $V_{DS}$ fall time (the MOSFET switch-in time) is slowed down to $\approx 250ns$ and the rise time (switch-off) is $\approx 50ns$. The required resistor values can be calculated using Equation \ref{eq:i_Gsink_per_fet_1} through Equation !!!. \\

To get the resistor values, first, the sink current must be calculated to draw the gate charge in the required switch-off time for the individual MOSFETs. 

    \begin{equation}
        i_{Gsink/FET} = \bigg( \frac{Q_{GD}}{t_{sw{\_}off}} \bigg)
        \label{eq:i_Gsink_per_fet_1}
    \end{equation}
    
    where
    
    \begin{itemize}
        \item $i_{Gsink/FET}$ is the required switch-off current
        \item $Q_{GD}$ is the gate-drain charge of the MOSFET
        \item $t_{sw{\_}off}$ is the desired switch-off time
    \end{itemize}
    
    \begin{equation}
        i_{Gsink/FET} = \bigg( \frac{V_{Miller}}{R_{G{\_}FET} + R_{G{\_}FET{\_}int}} \bigg)
        \label{eq:i_Gsink_per_fet_2}
    \end{equation} \\
    
    Rearranging Equation \ref{eq:i_Gsink_per_fet_2} yields the following:
    
    \begin{equation}
        R_{G{\_}FET} = \bigg( \frac{V_{Miller}}{i_{Gsink/FET}} - R_{G{\_}FET{\_}int} \bigg)
        \label{eq:i_Gsink_per_fet_2}
    \end{equation}
    
    where
    
    \begin{itemize}
        \item $R_{G{\_}FET}$ is the resistor for the individual MOSFETs
        \item $V_{Miller}$ is the Miller-plateau of the MOSFET
        \item $R_{G{\_}FET{\_}int}$ is the internal gate-source resistance of the MOSFET
    \end{itemize}
    
    Now the switch-on current and common resistance can be calculated:
    
    Taking after Equation \ref{eq:i_Gsink_per_fet_1}:

    \begin{equation}
        i_{Gsource/FET} = \bigg( \frac{Q_{GD}}{t_{sw{\_}on}} \bigg)
        \label{eq:i_Gsource_per_fet_1}
    \end{equation}    
    
    
    
    \begin{equation}
        i_{Gsource/FET} = \frac{\frac{V_G - V_{Miller}}{R_{pullup} + R_G}}{N}
        \label{eq:i_Gsoursce_per_fet_2}
    \end{equation}
    
    and
    
    \begin{equation}
        R_G = R_{G{\_}common} + \frac{R_{G{\_}FET} + R_{G{\_}FET{\_}int}}{N}
        \label{eq:R_G}
    \end{equation}
    
    Substituting Equation \ref{eq:R_G} into Equation \ref{eq:i_Gsoursce_per_fet_2} and rearranging them yields the following:
    
    \begin{equation}
        R_{G{\_}common} = \frac{i_{Gsource/FET} \cdot N}{V_G - V_{Miller}} - R_{pullup} - \frac{R_{G{\_}FET} + R_{G{\_}FET{\_}int}}{N}
        \label{eq:R_G_common}
    \end{equation}
    
    It is easy to see that when $N = 2$, the total current to be supplied and drawn by the driver is double of the indivdual MOSFETs.
    
    The following variables and their values, obtained from their respective datasheets or calculated, were used:
    
    \begin{itemize}
        \item $i_{Gsource/FET} = 136 mA$ is the required switch-on current per MOSFET
        \item $i_{Gsink/FET} = 680 mA$ is the required switch-off current per MOSFET
        \item $Q_{GD} = 34 nC$ is the gate-drain charge of the MOSFET
        \item $t_{sw{\_}on} = 250 ns$ is the desired switch-on time
        \item $V_G = 12V$ is the gate voltage applied
        \item $V_{Miller} = 4.4 V$ is the Miller-plateau of the MOSFET
        \item $R_{pullup} = 2.6 \ohm$ is the gate driver IC internal pullup resistance
        \item $R_G = 25.34 \ohm$ is the equivalent resistance seen by the gate driver
        \item $R_{G{\_}common} = 22.1 \ohm$ is the common resistance value
        \item $R_{G{\_}FET} = 5.17$ is the resistor for the individual MOSFETs
        \item $R_{G{\_}FET{\_}int} = 1.7 \ohm$ is the internal gate-source resistance of the MOSFET
        \item $N = 2$ is the number of parallel MOSFETs
    \end{itemize}
    
    The diode parallel to the common resistor 