\section{Abstract}
The project goes through the steps needed to design a three-phase inverter with a build-in motor controller for an electric go kart.
% The motor is a PMAC motor which have high torque and efficiency but the control is more complex than other motor types.
The inverter is designed to run the \textit{ME1117 Brushless PMAC motor} at 4.5kW continuous and 14kW peak. The inverter hardware consist of four separate PCB's. A main interface board that was provided as material for the project, an analog interface board that handles the analog signals, a power board on an aluminum PCB for each phase handling the four SMD MOSFET's for each leg and lastly the driver boards driving each power board. 

The inverter is controlled by a Xilinx Zybo Zynq-7000 ARM/FPGA Training Board. The controller measures rotor angle, phase currents and torque pedal position and controls the motor with Field Orientated Control to output the requested torque. 
At low speeds a more accurate rotor angle than the one out of the encoder is approximated with linear interpolation to make the motor run smoother.

% In the end was construction not finished do to time constrains. That leaves the project without conclusive evidence that the project was a success. The project has further discussion going into depths of what factors could have been done differently.