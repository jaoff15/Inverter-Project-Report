\section{Discussion}
\label{sec:discussion}

The section will discuss the few test result gathered throughout the project but mostly go into detail with areas where problems might come up.
A crucial factor of the project was that the inverter never was build due to time constraints and therefore it was never tested. Which results in some solutions only being backed up by calculations and simulations and not physical tests.
% This reason is mainly theoretical and not that make tests to back it up.

% It could have lead to better understanding of the current setup and potential problems when implementing it in a physical environment. 
% Especially the lack of testing and tweaking parameters on the motor control is a big setback. 

First the discussion will go through the possible problems with the hardware. Then the control and thermal characteristics are discussed and lastly the embedded system.




% \subsection{Components}
% When it come to any component the correct one isn't picked on first try. The pick for the transistor and the gate driver, was the fourth suggestion and need to be picked in order to continue work on the PCB design. Since the construction and testing wasn't completed, more time could have been allocated, if an earlier decision had been made. \\

% To more precisely comment, a better way to compare the different components would have given a better result. Such a comparison could have been done in a Matlab script, that runs the choices through the same calculations, and puts out the important factors up against each others. \\




\subsection{Power board}
% Placing bulk capacitors on the busbars is rather difficult without burning the caps and reflowing the whole assembly. It should be substituted with a more prototype-friendly solution. 
\subsubsection{Construction}
Constructing the inverter in the intended way might turn out difficult. While mounting the bus bars across all three phase PCB's and mounting the capacitors directly on top of the bus bar might be a good idea electrically, it can turn out to be near impossible in practice.

The bus bars are made of solid copper which have a high thermal conductivity and therefore soldering anything directly to the bars are difficult. Heating up all the PCB's and the bus bar in an oven or heating plate is the only option. 
Getting all phase PCB's to align and solder correctly to the bus bars requires high precision and it might not be feasibly to use the same PCB's for a round two if something goes wrong.
On top of mounting the bus bars to the PCB's the capacitors has to be mounted at the same time. 

A compromise with a bit worse electrical characteristics but better construction might need to be used.

If the budget would allow, a larger heat sink would be a better choice. Being able to fix the output cables to the heat sink could help with improving the mechanical stability of the assembly and prevent breaking of the output screw terminals.

\subsubsection{Capacitors}
Finding out if the capacitance built in is enough to smooth out everything is still left to be tested.

\subsection{Driver board}
A mistake has caused only one of the split gate resistors being placed in the schematic. A resistor closer to the gate should also have been placed. Using an SMD diode could also lead to a better layout, but utilizing components from the stock was chosen to keep down the cost.

\subsection{Control}
A fully working model of the system was not achieved.
In some cases the model works as expected one of these being when the load on the motor increases resulting in an increase in amplitude of the three phase currents. As can be seen on figure \ref{fig:iabc} in section \ref{sec:simulation}.

When the load of the motor increases the torque of the motor drops, and then stabilizes at zero again after around $40 ms$ to $50 ms$. When the load changes the torque momentarily drops the same amount and then returns back to zero. As can be seen on figure \ref{fig:torque} in section \ref{sec:simulation}.

On figure \ref{fig:speed} in section \ref{sec:simulation} the speed of the motor can be seen. It can be seen that the speed is negative which is caused by the torque being divided by the inertia, before it is integrated into the speed. This results in the speed dropping when the load of the motor is changed.
Combining that with the speed not increasing fast enough between the acceleration drops. The speed should not be negative when the torque produced by the electrical field is more than the load of the motor.

Figure \ref{fig:idq} in section \ref{sec:simulation} shows the d- and q-currents. There it can be seen that the q-current is not going towards the set reference. It can also be seen that the d-current is going towards zero but slowly. This suggest that the PI-controller is not set correctly, or there is an error in the model. 


\subsection{Embedded software}
\subsubsection{PWM module}
As can be seen in the end of section \ref{sec:pwm} where the PWM modules are tested. The module work and it can produce a sine curve at the selected frequency. Three module instances can be used phase shifted $120^\circ$ resulting in a three-phase signal.

\subsubsection{Minimizing the deadtime in the PWM module}
The deadtime between the high-side and low-side PWM's are $\sim 53$ times longer than the turn off time of the transistors used. The deadtime is mainly limited by the resolution of the internal counter in the PWM module. By increasing the counter limits to $1000$ instead of $100$ the resolution would become $10$ times higher. The deadtime resolution would also become $10$ times higher which would bring the minimum deadtime down to $400ns$ only $\sim 5.3$ times larger than necessary.

\subsubsection{Linear interpolation}

The linear interpolation use a couple of assumptions and it is difficult to quantify the expected benefit from having it. As shown in section \ref{sec:linear_interpolation} a more precise rotor angle can be achieved for speeds less than $1185RPM$ for moments with low to no acceleration or deceleration. The error caused by acceleration and deceleration is not analyzed, only a static simulation has been made. To precisely determine how big of an advantage or disadvantage the interpolation can have in a real scenario, the algorithm would need to be dynamically tested.

\subsubsection{Waste of block RAM}
Section \ref{sec:com_pl_ps} describes how the communication of PWM thresholds are send from the processing system to the logic through a piece of block RAM. Only $0.08\%$ of the total block RAM is actually in use which is a terrible waste. The interface between the two systems should have been done using the AXI interface instead.


 
 