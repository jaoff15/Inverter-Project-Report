
\subsection{Clarke and Park Transformations}




% ************************** Clarke Transformation ***********************************
\subsubsection{Clarke Transformation}
The Clarke transformation converts three rotating phase currents to one rotating vector in the alpha-beta frame. The Clarke transformation is performed by using the two equations \ref{eq:clarke_transformation}.

\begin{equation}
    i_{\alpha} = \frac{2}{3} \cdot i_a - \frac{1}{3} \cdot i_b - \frac{1}{3} \cdot i_c
    , \hspace{1cm}
    i_{\beta} = 0 \cdot i_a + \frac{1}{\sqrt{3}} \cdot i_b - \frac{1}{\sqrt{3}} \cdot i_c
    \label{eq:clarke_transformation}
\end{equation}

It can also be written in matrix form as in equation \ref{eq:CtransMatrix}. 

\begin{equation}
    \centering
    \begin{bmatrix}
        i_{\alpha} \\ 
        i_{\beta}
    \end{bmatrix}
    =
    \begin{bmatrix}
        \frac{3}{2} & -\frac{1}{3} & -\frac{1}{3} \\
        0 & \frac{1}{\sqrt{3}} & -\frac{1}{\sqrt{3}} \\
    \end{bmatrix}
    \begin{bmatrix}
        i_{a} \\ 
        i_{b} \\ 
        i_{c}
    \end{bmatrix}
    \label{eq:CtransMatrix}
\end{equation}


% ************************** Park Transformation ***********************************
\subsubsection{Park Transform}
To convert the one rotating vector in the alpha-beta frame to a static vector in the rotating dp-frame, the Park transformation is used. The park transformation is performed with equation \ref{eq:park_transformation}.

\begin{equation}
    i_{d} = cos(\omega t) \cdot i_{\alpha} + sin(\omega t) \cdot i_{\beta}
    , \hspace{1cm}
    i_{q} = -sin(\omega t) \cdot i_{\alpha} + cos(\omega t) \cdot i_{\beta}
    \label{eq:park_transformation}
\end{equation}

$\omega t$ is the angle of the electrical field, based on the angle of the rotor.
The Park transformation can be seen on matrix form in equation \ref{eq:PtransMatrix}.


\begin{equation}
    \centering
    \begin{bmatrix}
        i_{d} \\ 
        i_{q}
    \end{bmatrix}
    =
    \begin{bmatrix}
       cos(\omega t) & sin(\omega t) \\
       -sin(\omega t) & cos(\omega t)
    \end{bmatrix}
    \begin{bmatrix}
        i_{\alpha} \\ 
        i_{\beta}
    \end{bmatrix}
    \label{eq:PtransMatrix}
\end{equation}





% ************************** Inverse Park Transformation ***********************************
\subsubsection{Inverse Park Transform}
Converts a static vector in a rotating dq-frame into a rotating vector in a alpha-beta frame.

\begin{equation}
    \centering
    \begin{bmatrix}
        i_{\alpha} \\ 
        i_{\beta}
    \end{bmatrix}
    =
    \begin{bmatrix}
       cos(\omega t) & -sin(\omega t) \\
       sin(\omega t) & cos(\omega t)
    \end{bmatrix}
    \begin{bmatrix}
        i_{d} \\ 
        i_{q}
    \end{bmatrix}
\end{equation}

Where
$\omega t = \varphi$

\begin{equation}
    i_{\alpha} = cos(\varphi) \cdot i_{d} - sin(\varphi) \cdot i_{q}
    , \hspace{1cm}
    i_{\beta} = sin(\varphi) \cdot i_{d} + cos(\varphi) \cdot i_{q}
    \label{eq:inverse_park_transformation}
\end{equation}

% ************************** Inverse Clarke Transformation ***********************************
\subsubsection{Inverse Clarke Transform}
Converts a rotating vector in the alpha-beta frame into three phase signals. 

\begin{equation}
    \centering
    \begin{bmatrix}
        i_{a} \\ 
        i_{b} \\ 
        i_{c}
    \end{bmatrix}
    =
    \begin{bmatrix}
        1 & 0 \\
        - \frac{1}{2} & \frac{\sqrt{3}}{2} \\
        - \frac{1}{2} & - \frac{\sqrt{3}}{2}
    \end{bmatrix}
    \begin{bmatrix}
        i_{\alpha} \\ 
        i_{\beta}
    \end{bmatrix}
\end{equation}


\begin{equation}
    i_{a} = i_{\alpha}
    , \hspace{1cm}
    i_{b} = -\frac{1}{2} \cdot i_{\alpha} + \frac{\sqrt{3}}{2} \cdot i_{\beta}
    , \hspace{1cm}
    i_{c} = -\frac{1}{2} \cdot i_{\alpha} - \frac{\sqrt{3}}{2} \cdot i_{\beta}
    \label{eq:inverse_clarke_transformation}
\end{equation}