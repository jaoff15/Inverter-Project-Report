\subsection{PI-controller}

To set the PI-controller the transfer functions for the d- and q-direction is determined, with the voltages as input and the current as output. This is done based on the equations for the voltages in the d- and q-direction, equation \ref{eq:d_direction} and \ref{eq:q_direction}.
Because a transfer function only can have one input and one output, the two equation, \ref{eq:d_direction} and \ref{eq:q_direction}, is shortened. The new equation can be seen in equation \ref{eq:d_direction3} and \ref{eq:q_direction3}.

\begin{equation}
    \label{eq:d_direction3}
    v_d = L_d \frac{d i_d}{dt} + R_s i_d
\end{equation}

\begin{equation}
    \label{eq:q_direction3}
    v_q = L_q \frac{d i_q}{dt} + R_s i_q
\end{equation}

As seen in the two equations, \ref{eq:d_direction3} and \ref{eq:q_direction3}, the parts in the equations depending on other variables than the current is neglected. 
The equations is Laplace transformed and converted to the transfer functions for the two systems.

\begin{equation}
    \label{eq:transfer_d}
    G_q = \frac{I_d(s)}{V_d(s)} = \frac{ \frac{1}{ \frac{L_d}{R_s} } }{ R_s \bigg(\frac{1}{ \frac{L_d}{R_s} } + s \bigg) }
\end{equation}

\begin{equation}
    \label{eq:transfer_q}
    G_q = \frac{I_q(s)}{V_q(s)} = \frac{ \frac{1}{ \frac{L_q}{R_s} } }{ R_s \bigg(\frac{1}{ \frac{L_q}{R_s} } + s \bigg) }
\end{equation}

Equation \ref{eq:transfer_d} and \ref{eq:transfer_q} is the transfer functions for the motor.

To set the PI-controller the cut off frequency is determined. The cut off frequency, $\omega_{cutoff}$, is determined from the time constant of the motor.

\begin{equation}
    \omega_{cutoff} = \frac{1}{\tau}
\end{equation}

$\tau$ is the 